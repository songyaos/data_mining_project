%%%%%%%%%%%%%%%%%%%%%%%%%%%%%%%%%%%%%%%%%
\documentclass[paper=a4, fontsize=15pt]{article} % A4 paper and 11pt font size

\usepackage{amsmath,amsfonts,amsthm} % Math packages
\usepackage{graphicx}
\usepackage[top=0.5in, bottom=1in, left=1in, right=1in]{geometry}
\usepackage{tabularx}


\pagestyle{plain} % Makes all pages in the document conform to the custom headers and footers

\title{	
Data Mining \\
Homework 2
}
\author{Yao Song\\301266041} % Your name
\date{\normalsize\today} % Today's date or a custom date

\begin{document}
\maketitle % Print the title
%----------------------------------------------------------------------------------------
%	PROBLEM 1
%----------------------------------------------------------------------------------------

\section*{MapReduce}
\subsection*{2.3.4}
mapper: The mapper will read in the bag and the query constraint. The mapper outputs each item which passes the constraint. 
The output key-value pair format is [(first name, last name),1], where key is a tuple containing first name and last name.

reducer: The reducer takes all the output from the mapper.
The reducer then combines and counts all the items with the same last name. Then the reducer will output the key-value pair [(first name, last name), count] for each distinct last name.

a sample input like the following:\\
Tom Lee\\
John Smith\\
Tom Lee\\
James Bond\\
Tom White\\
Tom White\\
Tom Blake\\
with query "first name  = Tom" will output something like this:\\
(Tom , Blake )	1\\
(Tom , White )	2\\
(Tom , Lee )	2

Python code to implement this process appended at last.


%%%%%%%%%%%%%%%%%%%%%%%%%%%%%%%%%%%%%%%%%%%%%%%%%%%%%%%%%%%%%%%%%%%%%%%%%%%%%%%%%%

\section*{Link Analysis}
\subsection*{5.4.2}
\subsubsection*{Trust Rank}
Assume $\beta = 0.8$. 
Since only node B is reliable, the teleport set is $S = \{B\}$ and we have $\boldsymbol{e}_s = [0,1,0,0]$. 
Then the vector $(1-\beta)\boldsymbol{e}_s/|S|$ has 0.2 for only second component and zero for other parts.
The iteration equation can be written as:\\
\begin{align}
v' 
&= \beta M v + (1-\beta)\boldsymbol{e}_s/|S|  \\
&= 
\begin{bmatrix}
       0 			 & \frac{2}{5} & \frac{4}{5}  	& 0          \\[0.3em]
       \frac{4}{15}  & 0           & 0 				& \frac{2}{5}					\\[0.3em]
       \frac{4}{15}  & 0		   & 0 				& \frac{2}{5}					\\[0.3em]
       \frac{4}{15}  & \frac{2}{5} & 0				& 0
\end{bmatrix}v 
+ 
\begin{bmatrix}
       0 			 \\[0.3em]
       \frac{1}{5}  \\[0.3em]
	   0  \\[0.3em]
       0
\end{bmatrix}
\end{align}

After calculation, 
the trust rank is $t =[\frac{198}{735},\frac{263}{735},\frac{116}{735},\frac{158}{735}  ]$
\subsubsection*{Spam Mass}
The page rank as given in the book is $r = [\frac{3}{9},\frac{2}{9},\frac{2}{9},\frac{2}{9}]$.\\
\\
So the spam mass is given by the formula $\frac{r-t}{r}$ is 
$[\frac{94}{490},\frac{-299}{490},\frac{142}{490},\frac{16}{490}]$.
A list of these vectors is given below:\\
\\
\begin{tabular}{l*{6}{c}r}
Node             & Page Rank & Trust Rank & Spam Mass  \\
\hline\\
A		            & $\frac{3}{9}$ & $\frac{198}{735}$ & $\frac{94}{490}$   \\[1em]
B		           	& $\frac{2}{9}$ & $\frac{263}{735}$ & $\frac{-299}{490}$  \\[1em]
C				    & $\frac{2}{9}$ & $\frac{116}{735}$ & $\frac{142}{490}$ \\[1em]
D				    & $\frac{2}{9}$ & $\frac{158}{735}$ & $\frac{16}{490}$ \\

\end{tabular}


\subsection*{5.5.1}
The link matrix of Fig. 5.1 in the book is given below:\\
\begin{align}
L = 
\begin{bmatrix}
       0 & 1 & 1  & 1   \\[0.3em]
       1  & 0 & 0 & 1	\\[0.3em]
	   1  & 0 & 0 & 0	\\[0.3em]
       0 & 1 & 1  & 0
\end{bmatrix}
\end{align}
After calculation, the limits of a and h are: 
\begin{align}
a = 
\begin{bmatrix}
       0.289    \\[0.3em]
       1.000 	\\[0.3em]
	   1.000 	\\[0.3em]
       0.813
\end{bmatrix},
h=
\begin{bmatrix}
       1.000    \\[0.3em]
       0.392 	\\[0.3em]
	   0.103 	\\[0.3em]
       0.711
\end{bmatrix}.
\end{align}





%%%%%%%%%%%%%%%%%%%%%%%%%%%%%%%%%%%%%%%%%%%%%%%%%%%%%%%%%%%%%%%%%%%%%%%%%%%%%%%%%%
\section*{Recommender System}
\subsection*{9.2.1}
\subsubsection*{a}
The cosine of the angel between $a$ and $b$ is :\\
\begin{align}
\frac{8.2008 + 160000\alpha^2 + 24\beta^2}
{\sqrt{9.3636 + 250000\alpha^2 + 36\beta^2} \sqrt{7.1824 + 102400\alpha^2 + 16\beta^2}}
\end{align}

The cosine of the angel between $a$ and $c$ is :\\
\begin{align}
\frac{8.9352 + 320000\alpha^2 + 36\beta^2}
{\sqrt{9.3636 + 250000\alpha^2 + 36\beta^2} \sqrt{8.5264 + 409600\alpha^2 + 36\beta^2}}
\end{align}

The cosine of the angel between $b$ and $c$ is :\\
\begin{align}
\frac{7.8256 + 204800\alpha^2 + 24\beta^2}
{\sqrt{7.1824 + 102400\alpha^2 + 16\beta^2} \sqrt{8.5264 + 409600\alpha^2 + 36\beta^2}}
\end{align}
\subsubsection*{b}
if $\alpha=1$ and  $\beta=1$, the data is listed below:\\
\begin{tabular}{|c|c|c|c|}
\hline 
x,y & a,b & a,c & b,c \\ 
\hline 
cos($\theta$) & 0.999997333284 & 0.999995343121 & 0.999987853375 \\ 
\hline 
$\theta(degrees)$  & 0.140345459308   &   0.181185239071    & 0.292152614283 \\
\hline
\end{tabular} 

\subsubsection*{c}
if $\alpha=0.01$ and  $\beta=0.5$, the data is listed below:\\
\begin{tabular}{|c|c|c|c|}
\hline 
x,y & a,b & a,c & b,c \\ 
\hline 
cos($\theta$) & 0.990881500541 & 0.991554714333 & 0.969177921994\\ 
\hline 
$\theta(degrees)$  & 7.74400574467  & 7.45370628496  & 15.8939985811\\
\hline
\end{tabular} 

\subsubsection*{d}
In this case, $\alpha=3.0/1460$ and  $\beta=3/16$, the data is listed below:\\
\begin{tabular}{|c|c|c|c|}
\hline 
x,y & a,b & a,c & b,c \\ 
\hline 
cos($\theta$) & 0.994390403954 &  0.995613999454 &  0.982246918405 \\ 
\hline 
$\theta(degrees)$ & 6.0718677355  &  5.37067531447 &  10.814438368\\
\hline
\end{tabular} 



\subsection*{9.2.3}
\subsubsection*{a}
The average rating of the user is $(4+2+5)/3 =3.67 $. 
So the normalized rating of the user is\\ $A:0.33,B:-1.67,C:1.33$
\subsubsection*{b}
The user profile is computed as follows:\\
for the processor speed: (3.06*4+2.68*2+2.92*5)/(4+2+5) = 2.92\\
for the disk size: (500*4+320*2+640*5)/(4+2+5) = 530\\
for the main memory size: (6*4+4*2+6*5)/(4+2+5) = 5\\
So the user profile is (2.92, 530, 5)



\subsection*{Q1}
The cosine similarity between Alice and other users are listed blow:\\
\begin{tabular}{|c|c|c|c|c|}
\hline 
• & u1 & u2 & u3 & u4 \\ 
\hline 
similarity & 0.76 & 0.99 & 0.91 & 0.31 \\ 
\hline 
\end{tabular}
%0.762456367879 0.992914564541 0.912280701754 0.314373041714
\\With the threshold 0.75, we know the neighbor are user1 and user2. We do not need to consider user3 because it does not rate item 5.

Thus the estimated rating of Alice on item 5 would be\\
$(5+4+4)/3.0 + (0.76*(3 - 11/4.0) + 0.99*(5 - 16/4.0))/(0.76+0.99) \approx 5$ \\
So Alice's predicated rating on item 5 is 5.




%%%%%%%%%%%%%%%%%%%%%%%%%%%%%%%%%%%%%%%%%%%%%%%%%%%%%%%%%%%%%%%%%%%%%%%%%%%%%%%%%%



\section*{Finding Similar Items}
\subsection*{3.3.3}
\subsubsection*{a}
The minhash signature is listed in the following table.\\
\begin{tabular}{|>{$}l<{$}|>{$}l<{$}|>{$}l<{$}|>{$}l<{$}|>{$}l<{$}|}
\hline 
 & s_1 & s_2 & s_3 & s_4 \\ 
\hline 
h_1 & 5 & 1 & 1 & 1 \\ 
\hline 
h_2 & 2 & 2 & 2 & 2 \\ 
\hline 
h_3 & 0 & 1 & 4 & 0 \\ 
\hline 
\end{tabular} 
\subsubsection*{b}
$h_3$ is a true permutation as shown in the following table.\\
%\begin{tabular}{|>{$}l<{$}|>{$}l<{$}|>{$}l<{$}|>{$}l<{$}|}
%\hline 
%element & h_1 & h_2 & h_3 \\ 
%\hline 
%0 & 1 & 2 & 2 \\ 
%\hline 
%1 & 3 & 5 & 1 \\ 
%\hline 
%2 & 5 & 2 & 0 \\ 
%\hline 
%3 & 1 & 5 & 5 \\ 
%\hline 
%4 & 3 & 2 & 4 \\ 
%\hline 
%5 & 5 & 5 & 3 \\ 
%\hline 
%\end{tabular} 
\begin{tabular}{l*{6}{c}r}
Element             & $h_1$ & $h_2$ & $h_3$  \\
\hline
0				  	& 1 & 2 & 2   \\
1		            & 3 & 5 & 1   \\
2		           	& 5 & 2 & 0  \\
3				    & 1 & 5 & 5 \\
4				    & 3 & 2 & 4 \\
5				    & 5 & 5 & 3 \\
\end{tabular}
\subsubsection*{c}
The comparison of estimated similarities and Jaccard Similarities are listed below.\\
\begin{tabular}{|>{$}l<{$}||c|c|}
\hline 
• & esimated sim & Jaccard sim \\ 
\hline 
s_1,s_2 & 1/3 & 2/6 \\ 
\hline 
s_1,s_3 & 1/3 & 2/6 \\ 
\hline 
s_1,s_4 & 2/3 & 3/6 \\ 
\hline 
s_2,s_3 & 2/3 & 2/6 \\ 
\hline 
s_2,s_4 & 2/3 & 3/6 \\ 
\hline 
s_3,s_4 & 2/3 & 3/6 \\ 
\hline 
\end{tabular} 

\subsection*{3.4.2}
The exact and estimated value of $s$ for each case are listed in the table below:\\
\begin{tabular}{|c|c|c|}
\hline 
• & exact & estimated \\ 
\hline 
(3,10) & 0.41 & 0.46 \\ 
\hline 
(6,20) & 0.57 & 0.61 \\ 
\hline 
(5,50) & 0.42 & 0.46 \\ 
\hline 
\end{tabular} 
\\We can observe that the exact value of $s$ is always smaller than the corresponding estimated one.
%%%%%%%%%%%%%%%%%%%%%%%%%%%%%%%%%%%%%%%%%%%%%%%%%%%%%%%%%%%%%%%%%%%%%%%%%%%%%%%%%%




%----------------------------------------------------------------------------------------

\end{document}