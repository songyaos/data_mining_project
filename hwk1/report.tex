%%%%%%%%%%%%%%%%%%%%%%%%%%%%%%%%%%%%%%%%%


\documentclass[paper=a4, fontsize=13pt]{article} % A4 paper and 11pt font size

\usepackage{amsmath,amsfonts,amsthm} % Math packages
\usepackage{graphicx}
\usepackage[top=0.5in, bottom=1in, left=0.5in, right=0.5in]{geometry}
\usepackage{tabularx}

%\usepackage{fancyhdr} % Custom headers and footers
\pagestyle{plain} % Makes all pages in the document conform to the custom headers and footers

\title{	
Data Mining \\
Homework 1
}

\author{Yao Song\\301266041} % Your name

\date{\normalsize\today} % Today's date or a custom date

\begin{document}

\maketitle % Print the title

%----------------------------------------------------------------------------------------
%	PROBLEM 1
%----------------------------------------------------------------------------------------

\section{Question 1}
\subsection{find the frequent item sets}
Due to the independence assumption, the support of an item sets $I$ with $k$ items can be written as 
$support(I)  = \prod_{i=1}^k P(i)$,where $support(\ast)$ is the support of $\ast$ and $P(\ast)$ is the probability of $\ast$.
We need to count each item set whose support is no less than $1\%$.
The problem turns out to be finding all groups of numbers from 1 to 10 with a product less than 100.
Certainly all the single items will be counted. Then all the pairs will also be counted.
Some triples and quadruples will also be counted. The detailed list of groups of items is as follows:\\
single items:\\
1 2 3 4 5 6 7 8 9 10\\
pairs:\\
(1, 2) (1, 3) (1, 4) (1, 5) (1, 6) (1, 7) (1, 8) (1, 9) (1, 10) (2, 3) (2, 4) (2, 5) (2, 6) (2, 7) (2, 8) (2, 9) (2, 10) (3, 4) (3, 5) (3, 6) (3, 7) (3, 8) (3, 9) (3, 10) (4, 5) (4, 6) (4, 7) (4, 8) (4, 9) (4, 10) (5, 6) (5, 7) (5, 8) (5, 9) (5, 10) (6, 7) (6, 8) (6, 9) (6, 10) (7, 8) (7, 9) (7, 10) (8, 9) (8, 10) (9, 10)\\
triples:\\
(1, 2, 3) (1, 2, 4) (1, 2, 5) (1, 2, 6) (1, 2, 7) (1, 2, 8) (1, 2, 9) (1, 2, 10) (1, 3, 4) (1, 3, 5) (1, 3, 6) (1, 3, 7) (1, 3, 8) (1, 3, 9) (1, 3, 10) (1, 4, 5) (1, 4, 6) (1, 4, 7) (1, 4, 8) (1, 4, 9) (1, 4, 10) (1, 5, 6) (1, 5, 7) (1, 5, 8) (1, 5, 9) (1, 5, 10) (1, 6, 7) (1, 6, 8) (1, 6, 9) (1, 6, 10) (1, 7, 8) (1, 7, 9) (1, 7, 10) (1, 8, 9) (1, 8, 10) (1, 9, 10) (2, 3, 4) (2, 3, 5) (2, 3, 6) (2, 3, 7) (2, 3, 8) (2, 3, 9) (2, 3, 10) (2, 4, 5) (2, 4, 6) (2, 4, 7) (2, 4, 8) (2, 4, 9) (2, 4, 10) (2, 5, 6) (2, 5, 7) (2, 5, 8) (2, 5, 9) (2, 6, 7) (2, 6, 8) (3, 4, 5) (3, 4, 6) (3, 4, 7) (3, 4, 8) (3, 5, 6)\\
quadruples:\\
(1, 2, 3, 4) (1, 2, 3, 5) (1, 2, 3, 6) (1, 2, 3, 7) (1, 2, 3, 8) (1, 2, 3, 9) (1, 2, 3, 10) (1, 2, 4, 5) (1, 2, 4, 6) (1, 2, 4, 7) (1, 2, 4, 8) (1, 2, 4, 9) (1, 2, 4, 10) (1, 2, 5, 6) (1, 2, 5, 7) (1, 2, 5, 8) (1, 2, 5, 9) (1, 2, 6, 7) (1, 2, 6, 8) (1, 3, 4, 5) (1, 3, 4, 6) (1, 3, 4, 7) (1, 3, 4, 8) (1, 3, 5, 6)


\subsection{prove there is no interesting association}
Let us assume $I$ is a set of items and $j$ is an item.
The association rule is $I \rightarrow j$.
The confidence of the association rule is $conf(I \rightarrow j)  = \frac{support(I, j)}{support(I)} = \frac{P(I,j)}{P(I)}$.
We know that all the baskets are independent of each other, no sets of items are correlated. 
So $conf(I \rightarrow j)  = \frac{P(I,j)}{P(I)} = P(j) = \frac{support(j)}{N} $, where $N$ is the number of baskets in total.

Thus, by the definition of interest: $Interest(j) = conf(I \rightarrow j) - \frac{support(j)}{N}$, we know $Interest(j) =0$ is always true.
We conclude that there are no interesting association rules in this data set.


\section{Question 2}
In pass 1 of the A-Priori algorithm, we need to count each of the one million items. This process needs $4\times10^6$ bytes of memory.

In the second pass, if we use the triangular matrix method, we need to count all the pairs made up from the $N$ frequent items.
That will take $4 \times \frac{N(N-1)}{2} = 2(N^2-N)$ bytes of memory.

If we use the triples method, we only need to count all the occurring pairs made up from the frequent items, that is $10^6 + M$ pairs according to assumptions 4,5 and 6 in the problem. And we need $12\times(10^6 + M)$ bytes of memory to count the pairs.

In conclusion, we need a total of $4\times10^6 + 2(N^2-N)$ bytes of main memory if we use the triangular matrix method in the second pass.
If we use the triples method, we need a total of $4\times10^6 + 12\times(10^6 + M) = 16\times10^6 + 12M$ bytes of main memory.
The minimum amount of memory needed is the minimum of the two values $min\{4\times10^6 + 2(N^2-N),16\times10^6 + 12M\}$, which depends on the value of $N$ and $M$.

\section{Question 3}
\subsection{compute support for each item and each pair of items}
The table for items:\\
\begin{tabular}{|c|c|c|c|c|c|c|}
\hline 
item & 1 & 2 & 3 & 4 & 5 & 6 \\ 
\hline 
support & 4 & 6 & 8 & 8 & 6 & 4 \\ 
\hline 
\end{tabular} 
\\
\\
The table for pairs:\\
\begin{tabular}{|c|c|c|c|c|c|c|c|c|c|c|c|c|c|c|c|}
\hline 
pair & 1,2 & 1,3 & 1,4 & 1,5 & 1,6 & 2,3 & 2,4 & 2,5 & 2,6 & 3,4 & 3,5 & 3,6 & 4,5 & 4,6 & 5,6 \\ 
\hline 
support & 2 & 3 & 2 & 1 & 0 & 3 & 4 & 2 & 1 & 4 & 4 & 2 & 3 & 3 & 2 \\ 
\hline 
\end{tabular} 

\subsection{which pairs hash to which buckets?}
\begin{tabularx}{1\textwidth}{|X|X|X|X|X|X|X|X|X|X|X|X|}
\hline 
buckets & 0 & 1 & 2 & 3 & 4 & 5 & 6 & 7 & 8 & 9 & 10 \\ 
\hline 
pairs & $\phi$ & (2,6) \newline (3,4) & (1,2)\newline(4,6) & (1,3) & (1,4)\newline(3,5) & (1,5) & (1,6)\newline(2,3) & (3,6) & (2,4)\newline(5,6) & (4,5) & (2,5) \\ 
\hline 
support & 0 & 5 & 5 & 3 & 6 & 1 & 3 & 2 & 6 & 3 & 2 \\ 
\hline
\end{tabularx} 

\subsection{which buckets are frequent?}
From the table above, we can observe buckets 1,2,4,8 are frequent.

\subsection{which pairs are counted on the second pass of the PCY algorithm?}
All items are frequent, only the pairs in the frequent buckets of pass 1 will be counted in the second pass of PCY algorithm.
The pairs are (2,6),(3,4),(1,2),(4,6),(1,4),(3,5),(2,4),(5,6).


\section{Question 4}
The second pass result is as follows:\\
\begin{tabularx}{1\textwidth}{|X|X|X|X|X|X|X|X|X|X|}
\hline 
buckets & 0 & 1 & 2 & 3 & 4 & 5 & 6 & 7 & 8  \\ 
\hline 
pairs & $\phi$ & (4,6) & (5,6) & (1,2) & $\phi$ & (1,4) & (2,4) & (3,4) & (2,6)\newline(3,5) \\ 
\hline 
support & 0 & 3 & 2 & 2 & 0 & 2 & 4 & 4 & 5  \\ 
\hline
\end{tabularx} 
The second pass reduced the set of candidate pairs. Now we only need to count 4 pairs.
They are (2,4) (2,6) (3,4) (3,5)

\newpage
\section{Question 5}
We know the CF =$75\%$, so z = 1.15. \\
The computation process is as follows:\\
\textbf{adoption of budget resolution =n:}\\
original subtree error:  2.80613948555\\
replaced leaf error:  2.73057782412\\
so we \textbf{prune} the subtree.\\
\\
\textbf{physician fee freeze=n:}\\
original subtree error:  4.61102506017\\
replaced leaf error:  2.96152268664\\
so we \textbf{prune} the subtree.
\\
\\
\textbf{duty free exports = n: }\\
original subtree error:  7.69930394053\\
replaced leaf error:  8.49222452865\\
so we do \textbf{not prune} the subtree
\\
\\
\textbf{synfuels corporation cutback =y:}\\
original subtree error:  9.06482071828\\
replaced leaf error:  10.6959586471\\
so we do \textbf{not prune} the subtree
\\
\\
\textbf{physician freeze=y:}\\
original subtree error:  15.7121952555\\
replaced leaf error:  15.1970073524\\
so we \textbf{prune} the subtree
\\
\\
\textbf{water project cost sharing =u:}\\
original subtree error:  2.74613354335\\
replaced leaf error:  3.22974452245\\
so we do \textbf{not prune} the subtree
\\
\\
\textbf{physician freeze = u:}\\
original subtree error:  3.74002739023\\
replaced leaf error:  4.72288283877\\
so we do \textbf{not prune} the subtree
\\
\\
We complete the pruning process now.
The final pruned tree is as follows:\\
\textbf{physician fee freeze = n: democrat(168/1)\\
physician fee freeze = y: republican(123/11)\\
physician fee freeze = u: \\
........	water project cost sharing = n: democrat(0)\\
........	water project cost sharing = y: democrat(4)\\
........	water project cost sharing = u:\\
................		mx missile = n: republican(0)\\
................		mx missile = y: democrat(3/1)\\
................		mx missile = u: republican(2)\\}

\newpage
\section{Question 6}
The algorithm will converge in 4 iterations.\\
For iteration 1:\\
\begin{tabular}{|c|c|c|c|c|}
\hline 
center & (2,10) & (5,8) & (1,2) & • \\ 
\hline 
point & dist1 & dist2 & dist3 & cluster \\ 
\hline 
(2,10) & 0 & 5 & 9 & 1 \\ 
\hline 
(2,5) & 5 & 6 & 4 & 3 \\ 
\hline 
(8,4) & 12 & 7 & 9 & 2 \\ 
\hline 
(5,8) & 5 & 0 & 10 & 2 \\ 
\hline 
(7,5) & 10 & 5 & 9 & 2 \\ 
\hline 
(6,4) & 10 & 5 & 7 & 2 \\ 
\hline 
(1,2) & 9 & 10 & 0 & 3 \\ 
\hline 
(4,9) & 3 & 2 & 10 & 2 \\ 
\hline 
\end{tabular} 
\\
\\
For iteration 2:\\
\begin{tabular}{|c|c|c|c|c|}
\hline 
center & (2,10) & (6,6) & (1.5,3.5) & • \\ 
\hline 
point & dist1 & dist2 & dist3 & cluster \\ 
\hline 
(2,10) & 0 & 8 & 7 & 1 \\ 
\hline 
(2,5) & 5 & 5 & 2 & 3 \\ 
\hline 
(8,4) & 12 & 4 & 7 & 2 \\ 
\hline 
(5,8) & 5 & 3 & 8 & 2 \\ 
\hline 
(7,5) & 10 & 2 & 7 & 2 \\ 
\hline 
(6,4) & 10 & 2 & 5 & 2 \\ 
\hline 
(1,2) & 9 & 9 & 2 & 3 \\ 
\hline 
(4,9) & 3 & 5 & 8 & 1 \\ 
\hline 
\end{tabular} 
\\
\\
For iteration 3:\\
\begin{tabular}{|c|c|c|c|c|}
\hline 
center & (3,9.5) & (6.5,5.25) & (1.5,3.5) & • \\ 
\hline 
point & dist1 & dist2 & dist3 & cluster \\ 
\hline 
(2,10) & 1.5 & 9.25 & 7 & 1 \\ 
\hline 
(2,5) & 5.5 & 4.75 & 2 & 3 \\ 
\hline 
(8,4) & 10.5 & 2.75 & 7 & 2 \\ 
\hline 
(5,8) & 3.5 & 4.25 & 8 & 1 \\ 
\hline 
(7,5) & 8.5 & 0.75 & 7 & 2 \\ 
\hline 
(6,4) & 8.5 & 1.75 & 5 & 2 \\ 
\hline 
(1,2) & 9.5 & 8.75 & 2 & 3 \\ 
\hline 
(4,9) & 1.5 & 6.25 & 8 & 1 \\ 
\hline 
\end{tabular} 
\\
\\
For iteration 4:\\
\begin{tabular}{|c|c|c|c|c|}
\hline 
center & (3.67,9) & (7,4.33) & (1.5,3.5) & • \\ 
\hline 
point & dist1 & dist2 & dist3 & cluster \\ 
\hline 
(2,10) & 2.67 & 10.67 & 7 & 1 \\ 
\hline 
(2,5) & 5.67 & 5.67 & 2 & 3 \\ 
\hline 
(8,4) & 9.33 & 1.33 & 7 & 2 \\ 
\hline 
(5,8) & 2.33 & 5.67 & 8 & 1 \\ 
\hline 
(7,5) & 7.33 & 0.67 & 7 & 2 \\ 
\hline 
(6,4) & 7.33 & 1.33 & 5 & 2 \\ 
\hline 
(1,2) & 9.67 & 8.33 & 2 & 3 \\ 
\hline 
(4,9) & 0.33 & 7.67 & 8 & 1 \\ 
\hline 
\end{tabular} 
\\
\\

%----------------------------------------------------------------------------------------

\end{document}
